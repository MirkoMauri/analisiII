\documentclass[a4paper,12pt]{article}
\usepackage[utf8x]{inputenc}
\usepackage{amssymb, amsmath}
\newcommand{\ubar}{\underbar}
\usepackage{fullpage}

\begin{document}
\begin{titlepage}
\title{Dispensa per il secondo compitino di Analisi II}
\author{Riccardo Iaconelli}
% Remove command to get current date 
\maketitle
\end{titlepage}

\begin{titlepage}
\tableofcontents
\end{titlepage}

\section{Successioni e serie}

Sia $\{f_n\}$ una successione di funzioni a valori reali o complessi su $E$.
$\{f_n\}$ converge \textbf{puntualmente} se: $$\forall \varepsilon>0, \forall x \in E\ \ \exists N=N(\varepsilon, x):\ \forall n \geq N\ \forall x\in E\ \ |f_n(x) - f(x)| < \varepsilon$$ ove $$f(x):=\lim_{n\to+\infty} f_n(x)$$
\subsection{Criteri di convergenza}

\subsubsection{Successioni}
$\{f_n\}$ converge uniformemente se e solo se $\forall \varepsilon>0\ \ \exists N=N(\varepsilon):\ \forall n,m \geq N\ \forall x\in E$ si ha che $|f_n(x) - f_m(x)| < \varepsilon$.

La convergenza uniforme di una successione è condizione \textbf{necessaria e sufficiente} per verificare il criterio di Cauchy.

\subsubsection{Serie}

\paragraph{Condizione sufficiente 1}
Sia $\{f_n\}$ una successione di funzioni su $E$. Sia $$M_n:= \displaystyle\sup_{x\in E}|f_n(x)|$$ e $$\sum_{n=1}^{+\infty}M_n<+\infty$$
allora $\displaystyle\sum_{n=1}^{+\infty}f_n$ converge uniformemente.

\subsection{Come risolvere gli esercizi}
\subsubsection{Successioni}
\paragraph{Punto 0}
Cerca dominio, etc...
\paragraph{Convergenza puntuale}
Calcola limite...
\paragraph{Convergenza uniforme}
Sup di fn...

\subsubsection{Serie}

\section{Altri teoremini interessanti}
\paragraph{Successioni, continuità}
Sia $\{f_n\}$ una successione di funzioni continue definite in $E$ uniformemente convergenti a $f$. Allora $f$ è continua.
\paragraph{Convergenza uniforme e integrazione}
...



\section{Equazioni differenziali}


\subsection{Tipi di equazioni}
\subsubsection{LNOCC}
\subsubsection{LNOCC}
\subsubsection{LNOCC}

$$\underbar{y}'=A\underbar{y}+\underbar{b}(t)$$
Ove $A\in M_n(\mathbb{R})$ e $b$ continua, allora
$$\Psi^*(t)= \int_{t_0}^t e^{(t-s)A} \underbar{b}(s)ds$$
è una soluzione particolare dell'equazione sopra.
Per orttenere una soluzione più specifica, ad esempio ponendo una condizione iniziale quale
$$\underbar{y}(t_0)=\underbar{y}_0$$
Allora si ha che
$$\phi(t)=e^{(t-t_0)A}\underbar{y}_0+\Psi^*(t)$$
ovvero si somma alla soluzione generale del sistema omogeneo associato la soluzione particolare trovata in precedenza.
\subsubsection{Metodo di variazione delle costanti arbitrarie}


\subsubsection{LNOCC}
\subsubsection{LNOCC}
\subsubsection{LNOCC}
\subsubsection{On heavens door...}
\subsection{Equazioni particolari}
\subsubsection{Equazioni lineari}
Con un'equazione di forma:
$$y'(t) = P(t)y+Q(t)$$
Allora
$$y(t)=e^{- \displaystyle\int P(t)dt}\left(k + \int Qe^{-\displaystyle\int P(t)dt}dt\right)$$
\subsubsection{Equazione di Bernoulli}
Con un'equazione di forma:
$$y'(t) = P(t)y+Q(t)$$
con $\alpha \neq 0, 1$. Allora, si impone:
$$ z(t)=y(t)^{1-\alpha} $$
si risolve l'equazione lineare in $z$: $$z'(t)=(1-\alpha)P(t)z(t)+Q(t)$$ e, trovata $z(t)$ con la formula sopra, si ricava $y(t)$.

\subsubsection{Equazione di Riccati}
$$y'(t) = P(t)y^2+Q(t)y+R(t)$$
Conoscendo una soluzione particolare $\Psi(t)$, allora applico la sostituzione
$$y(t)=\Psi(t)+\dfrac{1}{z(t)}$$
e con un po' di passaggi si dimostra che alla fine
$$z'(t)= -\left[2\Psi(t)P(t)+Q(t)\right]z(t)-P(t)$$
che è lineare.
\subsection{Variabili separabili}
Se invece
$$y'(t) = f(t)g(y)$$
Allora ovviamente si risolve ricordando che
$$\int\dfrac{dy}{g(y)}=\int f(t)dt + c$$
\subsection{Equazioni omogenee}
$$y'(t) = f\left(\dfrac{y(t)}{t}\right)$$
Allora si sostituisce
$$z=\dfrac{y}{t}$$
da cui $$z'(t) = \dfrac{dz}{dt} = \dfrac{f(z)-z}{t}$$
e dunque
$$\int\dfrac{dz}{f(z)-z}=\int \dfrac{dt}{t} + c$$
sicché
$$\int\dfrac{dz}{f(z)-z} = \log|t| + c$$
\subsubsection{Equazioni lineari di ordine $n$}
...

\section{Altro}
\subsection{Norme}
\paragraph{Norma 1}
La norma 1 è definita come:
$$||f||_1 := \int_a^b|f|$$y

\end{document}

